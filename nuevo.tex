\documentclass{article}
\usepackage[utf8]{inputenc}

\title{Solución de Taller}
\author{Alejandro Orjuela }
\date{Agosto 2020}

\usepackage{natbib}
\usepackage{graphicx}
\usepackage{multicol}
\usepackage{graphicx}
\graphicspath{ {C:/Users/Alejandro/Picture/}}
\usepackage{float}
\begin{document}
\maketitle

\begin{multicols}{2}
\section{Introducción}
 En el siguiente documento podremos responder algunas preguntas del taller propuesto  en la clase de teoría general de sistemas, en el cual nos veremos enfocados en preguntas de algunas lecturas.


\includegraphics[scale=0.15]{../../Pictures/lider.jpg} 


\section{Preguntas de lectura}

\textbf{Trabajo, Lectura Obligatoria Nro I.}
Después de leer, debe comentar las preguntas existentes:

\begin{enumerate}
\item Los trabajadores expresaron que los gerentes no los escuchaban o tomaban ventaja de sus sugerencias
\item  Los trabajadores no eran empoderados para hacer los cambios para los cuales tenían las habilidades y los recursos necesarios
\item  La alta gerencia era dependiente para la toma de decisiones de sus ejecutivos corporativos, los que se encontraban a tres mil millas de distancia.
\end{enumerate}


\textbf{Lectura Obligatoria Nro II.}
 Después de leer, debe comentar las preguntas existentes:
 

  \begin{enumerate}
\item  Quien es el líder usualmente no tiene la respuesta, explique su respuesta:
\item    El rol del líder es hacer que los seguidores asuman su responsabilidad (una tarea poco popular), por ejemplo quienes creen ustedes:
\item   Que las soluciones para lograr progresos en el desafío pueden tomar bastante tiempo para ser construidas y en cualquier caso sólo podrán aspirar a que sea la más los individualistas que tienen una baja adhesión a las reglas y roles y una baja adhesión los grupos, explique cuales son los individualistas.
\item   Los fatalistas que son personas con una alta adhesión a las reglas y una baja adhesión a los grupo
\item   Los igualitarios que se caracterizan por una alta adhesión al grupo y una baja adhesión a las reglas y role
\item   Los jerarquistas (hierarchist en inglés) con una alta adhesión a las reglas y roles y una alta orientación al grupo.
\end{enumerate}

 
\textbf{Lectura (Sálvese quien pueda de Andrés Oppenheimer).}
\begin{enumerate}
\item Describa que son los tecno-Optimistas y tecno-Negativistas y sus características.
\item  Hasta donde lleva el libro
\end{enumerate}

\section{Respuestas de preguntas taller}

\textbf{Trabajo, Lectura Obligatoria Nro I.}

\begin{enumerate}
\item En muchas empresas los gerentes se toman algunas atribuciones que no deberian, en ello toman las ideas de sus empleados y las hacen pasar como suyas, haciendo que la recompensa sea para ellos y no para sus empleados, tal vez para ganar más, en algunas otras nisiquiera les ponen atencion porque creen que ellos no saben del tema y así su palabra queda en el olvido.
\item  Se cree que los trabajadores al tener las habilidades y conocer su trabajo podrian cambiar para bien la forma de trabajo, pero los altos mandos creen que sus capacidades son inferiores, por ello no deben tenerse en cuenta.
\item  Se dice que se necesita la palabra total de la gerencia para llevar los cambios de la empresa, ya que estos la conocen total mente, pero en casos se necesita tomar decisiones rapidas y la gerencia se demora en estos casos.
\end{enumerate}

\textbf{Lectura Obligatoria Nro II.}
 
\begin{enumerate}
\item  Un lider no puede tener la respuesta absoluta ya que como persona, puede estar equivocado con lo que esta realizando, por ende necesita colaboracion de sus subordinados
\item   Un buen lider encamina a sus seguidores a un buen futuro donde debe enseñarle las bases de como lograrlo y ayudandolos a mejorar
\item   Los individualistas son aquellos que toman propias decisiones basados en sus experiencias y conocimientos por ellos tienden a soluciones un problema rapidamente o encontrar una solucion logica.
\item  Se basan en lo que esta reglamentado donde imponen la reglamentacion como un codigo pues creen en el determinismo de los acontecimientos, dirigidos por causas independientes de la voluntad humana.
\item  Los igualitarios creen que todo debe ser valanceado, la igualdad se refiere a imparcialidad y no discriminación, a considerar los intereses de todos por igual, creando una regla parcial para todos.
\item   Los jerarquistas tieneden a tener estructuras que se establecen en orden a su criterio de subordinación entre personas, animales, valores y dignidades. Tal criterio puede ser superioridad, inferioridad, anterioridad, posterioridad, etc.
\end{enumerate}


\textbf{Lectura (Sálvese quien pueda de Andrés Oppenheimer).}
\begin{enumerate}
\item Para los tecno-pesimistas, la velocidad del cambio reduce la capacidad de adaptación y las nuevas empresas tecnológicas no son empleadoras masivas, además de que la amplitud del proceso a todos los sectores de actividad no permitirá el trasvase de desocupados de unos sectores a otros, conduciendo a una mayor brecha social y una polarización del empleo (el llamado modelo laboral en forma de reloj de arena).

Para los tecno-optimistas, la tecnología ha ayudado a que, en los últimos 30 años, se haya
reducido a la mitad el número de habitantes que viven en extrema pobreza; en que los vaticinios del pasado con respecto al impacto negativo de la tecnología no se hayan cumplido (como ha ocurrido con el sector del automóvil frente al caballo o la industria textil); que se puede trabajar menos horas y que, incluso, las máquinas podrían ayudar, en un futuro más lejano, a que logremos el “pleno desempleo” y que el trabajo humano sea erradicado.

\item  Hace pensar el sentido de a que parrte la humanidad puede llegar en un futuro, llevado por la ayuda de las computadoras con inteligencia artificial, el libro es totalmente analítico y se rige por estudios, conceptos de importantes personalidades,  y hechos reales que suceden en la actualidad y que el mismo Oppenhimer va a constatar.
\end{enumerate}


\section{Conclusion}
Quiero concluir con algo basico, de todo lo que nos rodea siempre abra un cambio, ya sea en lo personal o tecnologico, la humanidad siempre buscara la forma de ahorrar trabajo para si mismo, lo malo es que muchas personas no tendran oficio y no se sabe de que podrian vivir, ya que si han podido estudiar una carrera durante gran parte de su vida, podrian ser cambiados por una maquina dejandolos sin trabajo. El mundo esta en un cambio inimagiable pero debemos ser concientes en que estamos haciendo bien o que esta mal, puesto que cada vez somos mas egoistas y no nos preocupamos por lo que pueda pasarle a los demas.

\bibliographystyle{plain}
\bibliography{references}

\end{multicols}
\end{document}
