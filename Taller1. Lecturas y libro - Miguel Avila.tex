\documentclass[journal]{IEEEtran}

% *** GRAPHICS RELATED PACKAGES ***
%
\ifCLASSINFOpdf
  \usepackage[pdftex]{graphicx}
\else
  
\fi


\hyphenation{op-tical net-works semi-conduc-tor}

\begin{document}
%

\title{Taller 1. Lecturas y libro}

\author{Luis Miguel Avila}

% The paper headers
\markboth{Journal of \LaTeX\ // Teoria General de Sistemas}
{Shell \MakeLowercase{\textit{et al.}}: Bare Demo of IEEEtran.cls for IEEE Journals}


\maketitle

\begin{abstract}
La teoría general de sistemas nos ayuda en el análisis de problemas en cualquier situación siempre y cuando se pueda enfocar desde el ámbito sistemático. Podemos detallar esto en las dos primeras secciones del paper.\\
Luego nos enfocamos en el futuro de la tecnología que ventajas nos traerá y que nos costara, la privacidad y los empleos de miles de personas. Sobre este tema hay muchos matices que exploraremos en esta reflexión del libro "¡Sálvese quien pueda!" de Andres Oppenheimer
\end{abstract}


\begin{IEEEkeywords}
Tecnología, Futuro, Automatización, Robotica, Inteligencia Artificial.
\end{IEEEkeywords}

\IEEEpeerreviewmaketitle

\section{Lectura Obligatoria I}

En un sistema como una empresa todas las partes deben ser escuchadas hasta los empleados mas bajo nivel ya que ellos tienen otra perspectiva de las problemáticas conduciendo a una solución optima.\\Por este motivo los empleados deberán tener mecanismos que busque el empoderamiento de ellos para hacer valer sus propuestas ante la empresa.\\La alta gerencia no podía tomar decisiones eficientemente pues los ejecutivos se encontraban en otra localización para esto seria necesario darle cierta autonomía a la gerencia regional, ellos conocerán mejor los beneficios y debilidades de las decisiones de la parte regional.\\La teoría general de sistemas nos ayuda a encontrar los problemas del sistema pero ademas nos permite establecer los mecanismos para la solución adecuada

\section{Lectura Obligatoria II}

1) El líder usualmente no tiene la respuesta puesto que tiene que ponerse en los zapatos de todos sus empleados para poder entender las problemáticas que los aquejan a ellos y a la empresa, necesitara la ayuda y confianza de sus empleados.
2) El rol del líder es hacer que los seguidores asuman su responsabilidad se asemeja mucho a los lideres políticos del país quienes responsabilizan a los demás por sus fracasos pero cuando triunfa si sobresale su propio orgullo.
3) Las soluciones a las problemáticas suelen ser a muy largo plazo por lo que aquí intervienen grandes demandas de esfuerzo y recursos y para llegar a la solución mas apropiada o intentar acercarse a ello. 
4) Los individualistas pueden ser muy buenos trabajadores haciendo su propio trabajo pero a la hora de trabajar en grupo no son tan eficientes ya que no se ponen de acuerdo con sus similares ni con sus superiores. Les gusta seguir sus propias normas.
5) Las personas fatalistas son aquellas que se ardieren muy bien a las normas pero casi no le gustan trabajar en equipo, sin embargo cuando les toca lo harán y podrán sacar los proyectos adelante.
6) De acuerdo con la lectura los igualitarios que se caracterizan por una alta adhesión al grupo pero no me parece que tengan deficiencia al seguir las reglas y roles
7) Los jerarquistas son como los lideres se ajustan muy bien a las reglas y mantienen al grupo unido por una meta común en la empresa son de las personas mas importantes en el grupo. Cabe resaltar el concepto del liderazgo adaptativo debe cambiar de parecer si la situacion del equipo o del mercado a cambiado para ofrecer una respuesta acorde a la situación. 


%\subsubsection{Subsubsection Heading Here}
%Lorem ipsum dolor sit amet consectetur adipiscing elit venenatis facilisis, ultrices ad diam torquent scelerisque 

\section{Libro - Sálvese quien pueda}

\subsubsection{Tecno-Optimistas y Tecno-Negativistas}
Los optimistas tecnológicos son aquellas pernas quienes tienen una inclinación a pensar de manera positiva sobre las consecuencias de la tecnología, buscan argumentar que la los nuevos avances que se avecinan tendrán grandes beneficios. Sostienen que si habrá un desplazamiento laboral pero que se crearan muchos mas empleos en nuevas disciplinas que antes ni existían, un ejemplo de ello fue cuando llegaron los carros se quedaron sin trabajo las personas de los carruajes, de los establos, los amansadores entre otras sin embargo nacieron nuevos empleos en la fabricación, ensamblaje, distribución y venta de automóviles, adicionalmente aparecieron talleres mecánicos y gasolineras.\\Las bondades también estarán presentes en la economía pues replican que tendrá un gran crecimiento y que los productos serán cada vez baratos debido a la automatización cada vez mayor.\\Una idea tentadora que expresan es un mundo futuro sin trabajos, todo lo harán las maquinas por nosotros pero actualmente para nuestro pensamiento se ve muy difícil este destino, sus defensores afirman que anteriormente las clases sociales mas altas nunca trabajaban, si no que se dedicaban a la filosofía o solo disfrutaban de las artes, consideraban el trabajo como algo sucio de clase baja. Todo el trabajo lo realizaba su servidumbre de igual manera en el futuro los robots harían todo nuestro trabajo.

\begin{figure}[!b]
\centering
\includegraphics[width=2.5in]{imagenes/RobotvsHuman}
\caption{Humano contra robot}
\label{fig_sim}
\end{figure}

Por otra parte los negativistas vaticinan lo peor, los robots y la automatización de cada aspecto de nuestra de la vida eliminara empleos hasta en las áreas que parecen muy poco probable que sean desplazadas. Ya tenemos restaurantes que prescinden de los camareros luego de pronto quedaremos sin chefs, inteligencias artificiales capaces de ganarle a los campeones de ajedrez, go y scrabell.\\ Recientemente se supo de una IA generadora de texto llamada GPT-3 capaz de escribir artículos enteros sobre el tema que le demos de entrada, ¿ya no necesitaríamos periodistas?. Ademas podemos pedirle que traduzca el mencionado articulo a cualquier idioma. Pero lo realmente preocupante fue que le pueden pedir que programara cosas sencillas y lograba producir unos códigos funcionales, ni siquiera el trabajo de los programadores esta a salvo y cualquiera pensaría que siendo un trabajo mas conceptual seria mas difícil de replicar por una maquina.\\ Según un pesimista nadie esta seguro; diagnósticos médicos dados por una inteligencia artificial, calses virtuales dadas por una maquina, soldados robot con protocolos para detectar y disparar al enemigo, aviones no tripulados usados para bombardear, oficinistas sin empleos y la impresión 3D desplazaría a muchas personas de la construcción. De que van a vivir estas personas que ya no tendrán empleo, esta es la consigna de los anti-tecnológicos, ¿se podrán adaptar a tiempo?.

\subsubsection{Opinión del libro y relevancia en el camino profesional.}
Es un libro muy interesante me ha ayudado a tener una mejor visión y puntos de vista sobre la tecnología y por supuesto como nuestra carrera es netamente relacionada con ella, es de gran utilidad para saber entender los avances y como debemos estar preparados para adaptarnos a las nuevas tecnologías y cabe resaltar que debemos analizar como nos afectaría a los pisases en vía de desarrollo donde la educación no es tan accesible como en otros países.\\Le recomendaría leer este libro a cualquiera pero es casi una obligación leerlo si esta involucrado en el área de la tecnológica y la programación

\section{Conclusión}
En conclusión según mi propio criterio estamos lejos de una automatización tan acelerada aunque siempre es mejor estar prevenidos y estudiar cual podría ser el impacto verdadero.\\ Yo me inclinaría mas hacia el lado del positivismo, la tecnología a traído grandes avances y al final siempre se han creado muchos más empleos. Y lo más importante pese a cualquier cosa que hagamos la tecnología siempre seguirá avanzando, esa es su naturaleza como el tiburón que si no se mueve se muere, y si la tecnología se estanca el ser humano también.\\Para consolidar, la educación sera el pilar para adaptarnos en este mundo que cada vez se mueve más rápido

\begin{thebibliography}{1}

\bibitem{IEEEhowto:kopka}
Fig.1. human-vs-robot-09; Flickr\\https://www.flickr.com/photos/12211467@N02/6890018257/

\end{thebibliography}

\end{document}


